\documentclass[12pt,a4paper,twoside]{report}         
\usepackage{cs}              
\usepackage{times}
\usepackage{graphicx}
\usepackage{latexsym}
\usepackage{amsmath,amsbsy}
\usepackage{amssymb}
\usepackage[matrix,arrow]{xy}
\usepackage[T1]{fontenc}
\usepackage{ae,aecompl}
%\usepackage{shortcut} %definitii pentru diacritice; 
\usepackage{amstext}
\usepackage{graphics}
\usepackage[T1]{fontenc}
\usepackage{ae,aecompl}
\usepackage{algorithm}
%\usepackage{algorithmic}
\usepackage{color}
\usepackage{color}

% \mastersthesis
\diplomathesis
% \leftchapter
\centerchapter
% \rightchapter
\singlespace
% \oneandhalfspace
% \doublespace

\renewcommand{\thesisauthor}{Radu PETRIŞEL}    %% Your name.
\renewcommand{\thesismonth}{June}     %% Your month of graduation.
\renewcommand{\thesisyear}{2019}      %% Your year of graduation.
\renewcommand{\thesistitle}{GESTURE DETECTION IN VIRTUAL REALITY USING LEAPMOTION} 
\renewcommand{\thesissupervisor}{Assist. Prof. Dr. Eng. Adrian SABOU}
\newcommand{\department}{\bf FACULTY OF AUTOMATION AND COMPUTER SCIENCE\\
COMPUTER SCIENCE DEPARTMENT}
\newcommand{\thesis}{LICENSE THESIS}
\newcommand{\utcnlogo}{\includegraphics[width=15cm]{img/tucn.jpg}}

\newcommand{\uline}[1]{\rule[0pt]{#1}{0.4pt}}
%\renewcommand{\thesisdedication}{P\u{a}rin\c{t}ilor mei}

\begin{document}
%\frontmatter
%\pagestyle{headings}

\newenvironment{definition}[1][Defini\c{t}ie.]{\begin{trivlist}
\item[\hskip \labelsep {\bfseries #1}]}{\end{trivlist}}

%\thesistitle                    %% Generate the title page.
%\authordeclarationpage                %% Generate the declaration page.

\pagenumbering{Roman}
\setcounter{page}{1}

\begin{center}
\utcnlogo

\department

\vspace{4cm}

{\bf \thesistitle} %LICENSE THESIS TITLE}

\vspace{1.5cm}

\thesis
\vspace{6cm}

Graduate: {\bf \thesisauthor{}} 

Supervisor: {\bf \thesissupervisor{}}

\vspace{3cm}
{\bf \thesisyear}
\end{center}

\thispagestyle{empty}
\newpage

\begin{center}
\utcnlogo

\department

\end{center}
\vspace{0.5cm}

%\begin{small}
\begin{tabular}{p{7cm}p{8cm}}
 %\hspace{-1cm}& APPROVED,\\
 \hspace{-1cm}DEAN, & HEAD OF DEPARTMENT,\\
 \hspace{-1cm}{\bf Prof. dr. eng. Liviu MICLEA} & {\bf Prof. dr. eng. Rodica POTOLEA}\\  
\end{tabular}
 
\vspace{2cm}

\begin{center}
Graduate: {\bf \thesisauthor}

\vspace{1cm}

{\bf \thesistitle}
\end{center}

\vspace{1cm}

\begin{enumerate}
  \item {\bf Project proposal:} {\it A Reactive Programming oriented Unity asset for gesture detection using the LeapMotion controller}
  \item {\bf Project contents:} {\it (enumerate the main component parts) Presentation page, advisor's evaluation, title of chapter 1, title of chapter 2, ..., title of chapter n, bibliography, appendices.}
  \item {\bf Place of documentation:} {\it Technical University of Cluj-Napoca, Computer Science Department}
  \item {\bf Consultants:} \thesissupervisor{}
  \item {\bf Date of issue of the proposal:} November 1, 2018
  \item {\bf Date of  delivery:} June 14, 2019
\end{enumerate}

\vspace{1.2cm}

\hspace{6cm} Graduate: \uline{6cm} 

\vspace{0.5cm}
\hspace{6cm} Supervisor: \uline{6cm} 
%\end{small}

\thispagestyle{empty}


\newpage
$ $
%\begin{center}
%\utcnlogo

%\department
%\end{center}

\thispagestyle{empty}
\newpage

\begin{center}
\utcnlogo

\department
\end{center}

\vspace{0.5cm}

\begin{center}
{\bf
Declara\c{t}ie pe proprie r\u{a}spundere privind\\ 
autenticitatea lucr\u{a}rii de licen\c{t}\u{a}}
\end{center}
\vspace{1cm}



Subsemnatul(a) \\
\uline{14.8cm}, 
legitimat(\u{a}) cu \uline{4cm} seria \uline{3cm} nr. \uline{4cm}\\
CNP \uline{9cm}, autorul lucr\u{a}rii \uline{2.8cm}\\
\uline{16cm}\\
\uline{16cm}\\
elaborat\u{a} \^{\i}n vederea sus\c{t}inerii examenului de finalizare a studiilor de licen\c{t}\u{a} la Facultatea de Automatic\u{a} \c{s}i Calculatoare, Specializarea \uline{7cm} din cadrul Universit\u{a}\c{t}ii Tehnice din Cluj-Napoca, sesiunea \uline{4cm} a anului universitar \uline{3cm}, declar pe proprie r\u{a}spundere, c\u{a} aceast\u{a} lucrare este rezultatul propriei activit\u{a}\c{t}i intelectuale, pe baza cercet\u{a}rilor mele \c{s}i pe baza informa\c{t}iilor ob\c{t}inute din surse care au fost citate, \^{\i}n textul lucr\u{a}rii \c{s}i \^{\i}n bibliografie.

Declar, c\u{a} aceast\u{a} lucrare nu con\c{t}ine por\c{t}iuni plagiate, iar sursele bibliografice au fost folosite cu 
respectarea legisla\c{t}iei rom\^{a}ne \c{s}i a conven\c{t}iilor interna\c{t}ionale privind drepturile de autor.

Declar, de asemenea, c\u{a} aceast\u{a} lucrare nu a mai fost prezentat\u{a} \^{\i}n fa\c{t}a unei alte comisii de examen de licen\c{t}\u{a}.

\^{I}n cazul constat\u{a}rii ulterioare a unor declara\c{t}ii false, voi suporta sanc\c{t}iunile administrative, respectiv, \emph{anularea examenului de licen\c{t}\u{a}}.

\vspace{1.5cm}

Data \hspace{8cm} Nume, Prenume

\vspace{0.5cm}

\uline{3cm} \hspace{5cm} \uline{5cm}

\vspace{0.5cm}
\hspace{9.4cm}Semn\u{a}tura

\thispagestyle{empty}

\newpage


%\listoftables
%\listoffigures

%\clearpage 
%\newpage

%\begin{comment}
%\include{guideline} 
%\end{comment}

\tableofcontents
\newpage

\pagenumbering{arabic}
\setcounter{page}{1}


\chapter{Introduction - Project Context}
\pagestyle{headings}

Virtual Reality is an experience that has gained huge popularity in the recent years. Because of this new means of interaction with this virtual world are needed and they should feel as natural as possible. Ergo, hand tracking and gesture detection is a "must have" for modern VR applications.

\section{Virtual reality}
The term "virtual" began its life in the late 1400s, meaning "being something in essence or effect, though not actually or in fact" \cite{Virtual}, but, in the IT context, the word has the meaning "not physically existing but made to appear by software" \cite{Virtual}. The original use of the phrase "virtual reality" is found in French playwright' Antonin Artaud collection of essays \textit{Le Théâtre et son double}, first published in 1938 \cite{TheatreAndItsDouble}.

\subsection{History}
The precise roots of virtual reality are challenged, partially because of how hard it was to formulate a definition of an alternate reality notion. In 1968, Ivan Sutherland created what was widely regarded as the first head-mounted display system for use in immersive simulation applications, with the help of his students. In the next two decades, VR devices were mainly used for medical, automobile industry design, militry training and flight simulation purposes.

The 1990s saw the first commercially extensive release of consumer headsets, notably \textit{Sega VR} (1991) and \textit{Sega VR-1} (1994) launched by Sega, and \textit{Nintendo's Virtual Boy} (1995). The 2000s were a period of comparative indifference from the public and investment towards VR techniques available on the market. Google launched \textit{Street View} in 2007, a service that offers panoramic views of a growing amount of global locations such as highways, indoor houses and rural regions, which also integrates a stereoscopic 3D mode as of 2010.



\begin{table}[ht]
\caption{Nonlinear Model Results}
\centering                          % tabel centrat 
\begin{tabular}{|c|c|c|c|}          % 4 coloane centrate 
\hline\hline                        % linie orizontala dubla
Case & Method\#1 & Method\#2 & Method\#3 \\ [0.5ex]   % inserare tabel
%heading
\hline                              % linie orizontal simpla
1 & 50 & 837 & 970 \\               % corpul tabelului 
2 & 47 & 877 & 230 \\
3 & 31 & 25 & 415 \\[1ex]           % [1ex] adds vertical space
\hline                              
\end{tabular}
  % titlul tabelului
\label{table:nonlin}                % \label{table:nonlin} introduce eticheta folosita pentru referirea tabelului in text; referirea in text se va face cu \ref{table:nonlin}
\end{table}

\begin{figure}[ht]
    \centering
\includegraphics[]{img/test.jpg}
    \caption{The figure`s name}
    \label{fig:imag}
\end{figure}

\chapter{Project Objectives and Specifications}


Describe the proper theme (as a research/design proposal, clearly formulated, with clear objectives, and some explanatory figures).

Stretches over about 10\% of the paper.

\section{Title}
\section{Other title}

\chapter{Bibliographic research}


Bibliographic research has as an objective the establishment of the references for the project, within the project domain/thematic. While writing this chapter (in general the whole document), the author will consider the knowledge accumulated from several dedicated disciplines in the second semester, 4$^{th}$ year (Project Elaboration Methodology, etc.), and other disciplines that are relevant to the project theme.

Represents about 15\% of the paper.

Each reference must be cited within the document text, see example below (depending on the project theme, the presentation of a method/application can vary).


This section includes citations for conferences or workshop~\cite{BellucciLZ04}, journals~\cite{AntoniouSBDB07}, 
and books~\cite{russell1995artificial}. 

In paper~\cite{AntoniouSBDB07} the authors present a detection system for moving obstacles based on stereovision and ego motion estimation. 
The method is ... {\it discus the algorithms, data structures, functionality, specific aspects related to the project theme, etc.}... Discussion: {\it pros and cons}.

In chapter~\ref{ch:analysis} of~\cite{strunk}, the {\it similar-to-my-project-theme algorithm} is presented, with the following features ...


\section{Title}
\section{Other title}


\chapter{Analysis and Theoretical Foundation}
\label{ch:analysis}

Together with the next chapter takes about 60\% of the whole paper

The purpose of this chapter is to explain the operating principles of the implemented application.
Here you write about your solution from a theory standpoint - i.e. you explain it and you demonstrate its theoretical properties/value, e.g.:
\begin{itemize}
 \item used or proposed algorithms
 \item used protocols
 \item abstract models
 \item logic explanations/arguments concerning the chosen solution
 \item logic and functional structure of the application, etc.
\end{itemize}

{\color{red} YOU DO NOT write about implementation.

YOU DO NOT copy/paste info on technologies from various sources and others alike, which do not pertain to your project.
}

\section{Title}
\section{Other title}


\chapter{Detailed Design and Implementation}

Together with the previous chapter takes about 60\% of the paper.

The purpose of this chapter is to document the developed application such a way that it can be maintained and developed later. A reader should be able (from what you have written here) to identify the main functions of the application.

The chapter should contain (but not limited to):
\begin{itemize}
 \item a general application sketch/scheme,
\item a description of every component implemented, at module level,
\item class diagrams, important classes and methods from key classes.
\end{itemize}

\chapter{Testing and Validation}

About 5\% of the paper
\section{Title}
\section{Other title}

\chapter{User's manual}

In the installation description section your should detail the hardware and software resources needed for installing and running the application, and a step by step description of how your application can be deployed/installed. An administrator should be able to perform the installation/deployment based on your instructions.

In the user manual section you describe how to use the application from the point of view of a user with no inside technical information; this should be done with screen shots and a stepwize explanation of the interaction. Based on user's manual, a person should be able to use your product.

\section{Title}
\section{Other title}

\chapter{Conclusions}

About. 5\% of the whole

Here your write:
\begin{itemize}
\item a summary of your contributions/achievements,
\item a critical analysis of the achieved results,
\item a description of the possibilities of improving/further development.
\end{itemize}
\section{Title}
\section{Other title}


%\addcontentsline {toc}{chapter}{Bibliography} 
\bibliographystyle{IEEEtran} 
\bibliography{thesis}%same file name as for .bib

\include{apendix}

\end{document}
